\documentclass{article}
\usepackage{amsmath}
\usepackage{amssymb}
\usepackage{bm}
\usepackage{color}
\usepackage[left = 3.5cm, right = 3.5cm, top = 3.0cm, bottom = 4.5cm]{geometry}
\usepackage{natbib}

\newcommand{\E}[1]{\mathrm{E}(#1)}

\title{POPAN with transience: Model description}
\author{}
\date{}

\begin{document}

\maketitle

\section{Standard POPAN models}

To fit a POPAN model by maximum likelihood, we need to specify the
probability of obtaining each observed capture history as a function
of the model parameters. A capture history is given by $\bm{\omega} =
(\omega_1, \ldots, \omega_k)$, where $\omega_t = 1\ (t = 1, \ldots,
k)$ indicates the individual was detected on occasion $t$ of a total
of $k$ occasions, and $\omega_t = 0$ indicates the individual was not
detected.

Our model has the following parameters:
\begin{itemize}
  \item $p_t$, detection probability. This parameter is the
    probability of an individual being detected on occasion $t$.
  \item $\phi_t$, survival probability. This parameter is the
    probability of an individual that is alive and in the population
    on occasion $t$ still being alive and in the population on
    occasion $t + 1$.
  \item $\psi_t$, per-capita recruitment rate. This is the expected
    number of individuals who enter the population on occasion $t +
    1$ per individual in the population on occasion $t$.
  \item $M$, the superpopulation size. Conceptually, this parameter is
    the number of individuals that are ever at risk of detection
    during the survey.
\end{itemize}
We can estimate these parameters separately for each occasion,
restrict them to be the same (e.g., $p_t = p$ for all $t$), or model
them with available covariates (e.g., $\mathrm{logit}(p_t) = \beta_0 +
\beta_1 x_t$).

Practioners often consider the recruitment parameter $p_{e, t}$ (the
entry proportions) rather than the per-capita recruitment rate
$\psi_t$ we do here, where $p_{e, t}$ is the proportion of the $M$
individuals in the superpopulation that were first available for
detection on occasion $t$. We use $\psi_t$ because it involves a
proportional relationship between population growth and population
size, which is more biologically realistic: all else being equal,
larger populations experience larger fluctuations in absolute size
over time. Nevertheless, we can calculate the proportion of the $M$
individuals entering the population on each occasion from our $\psi$
and $\phi$ parameters as follows.

Let $N_t$ be the number of individuals alive and in the population on
occasion $t$, so the expected number of animals joining the population
on occasion $t + 1$ is $\psi_t N_t$. The expected number of
individuals that survive between occasions $t$ and $t + 1$ is $\phi_t
N_t$, and therefore the expected total number of individuals in the
population on occasion $t + 1$, given the number in the population on
occation $t$ is $\E{N_{t + 1} \mid N_t} = \psi_t N_t + \phi_t N_t =
(\psi_t + \phi_t)N_t$ and by taking an average over $N_t$ we obtain
\begin{equation}
  \E{N_{t + 1}} = (\psi_t + \phi_t) \E{N_t}. \label{eq:e1}
\end{equation}
We can also express the expected population size at time $t + 1$ using
proportions of entry as follows:
\begin{equation}
  \E{N_{t + 1}} = \phi_t \E{N_t} + p_{e, t + 1} M. \label{eq:e2}
\end{equation}
By equating \eqref{eq:e1} and \eqref{eq:e2}, we have
\begin{align}
  \phi_t \E{N_t} + p_{e, t + 1} M &= (\psi_t + \phi_t) \E{N_t} \nonumber \\
  p_{e, t + 1} &= \frac{\psi_t \E{N_t}}{M}. \label{eq:petp1}
\end{align}

By definition, the expected number of individuals in the population on
the first occasion is $\E{N_1} = p_{e, 1} M$. If we knew $p_{e, 1}$,
then we could determine the remaining $p_e$ values by iteratively
applying Equations \eqref{eq:petp1} and \eqref{eq:e2}: we calculate
$\E{N_1} = p_{e, 1} M$, then $p_{e, 2}$ from $\E{N_1}$ using
\eqref{eq:petp1}, then $\E{N_2}$ from $p_{e, 2}$ using \eqref{eq:e2},
and so on. In practice, however, we don't know the correct value at
which to initialise $p_{e, 1}$, but we can leverage the fact that the
$p_e$ parameters must sum to one. We provisionally initialise at any
value (e.g., $p_{e, 1} = 1$), calculate the $p_e$ parameters using
this initialisaion, and finally rescale them to sum to one by
multiplying them by the same constant.

Once we have computed the $p_e$ parameters, we can use the standard
POPAN model likelihood [insert reference].

\section{POPAN models with transience}

Here we consider a variation of the standard POPAN model in which some
proportion of recruited individuals are transients. We let $\gamma_t$
be the probability that a randomly selected individual recruited on
occasion $t$ is a transient, and so the probability that they are
resident is $1 - \gamma_t$. Residents have the standard survival
probabilities (i.e., a resident in the population on occasion $t$
remains in the population on occasion $t + 1$ with probability
$\phi_t$). However, transients have survival probabilities of zero,
and are therefore only available for detection in the occasion on
which they were recruited. In other words, a transient recruited on
occasion $t$ is guaranteed to have $\omega_{t^{\prime}} = 0$ for all
$t^\prime \ne t$.

We do not observe whether a detected individual is a transient or a
resident. If we detect an individual on more than one occasion then we
know they are a resident, but we are unable to resolve the status of
an individual that was only detected on one occasion: they could be a
transient that was detected on the sole occasion they were in the
population, although they may still be a resident that evaded
detection on the other occasions they were in the population, and
indeed some residents are only in the population for a single
occasion: for any given resident recruited on time $t$, there is a
$(1 - \phi_t)$ probability that they fail to survive beyond this
single occasion. We can still use the standard POPAN likelihood as
long as we can calculate the entry probability parameters, $p_{e}$,
for our new model.

Let $T_t$ and $R_t$ be the number of transients and residents,
respectively, in the population on occasion $t$, so that $N_t = T_t +
R_t$. Similarly to the standard model described in the previous
section, we link recruitment on occasion $t + 1$ to the number of
residents in the population on occasion $t$. We assume that the
expected number of individuals (transients and residents combined)
recruited on occasion $t$ is $\psi_t R_t$, partitioned into transients
and residents by the proportions $\gamma_t$ and $(1 - \gamma_t)$,
respectively. Therefore, the expected number of residents in the
population on occasion $t + 1$, conditional on the number of residents
in the population on occasion $t$, is
\begin{equation}
\E{R_{t + 1} \mid R_t} = \phi_t R_t + (1 - \gamma_{t + 1}) \psi_t R_t, \nonumber
\end{equation}
where the first term is the expected number of surviving residents
from occasion $t$, and the second term is the number of newly
recruited residents. Taking the expectation of both sides over $R_t$,
we obtain
\begin{align}
  \E{R_{t + 1}} &= \phi_t \E{R_t} + (1 - \gamma_{t + 1}) \psi_t \E{R_t} \nonumber
  \\ &= \{ \phi_t + (1 - \gamma_{t + 1}) \psi_t\} \E{R_t}. \label{eq:ertp1}
\end{align}

Similarly, for transients, we have $\E{T_{t + 1} \mid R_t} = \gamma_{t
  + 1} \psi_t R_t$ and $\E{T_{t + 1}} = \gamma_{t + 1} \psi_t
\E{R_t}$, where we only have a term for recruited individuals, because
by definition transients do not survive between occasions. To find the
expected total population size on occation $t + 1$ we can sum the
expectations for transients and residents respectively, which gives
\begin{align}
  \E{N_{t + 1}} &= \E{T_{t + 1}} + \E{R_{t + 1}} \nonumber \\
  &= (\phi_t + \psi_t) \E{R_t}, \label{eq:entp1} \intertext{and so}
  \E{R_t} &= \frac{\E{N_{t + 1}}}{\phi_t + \psi_t}. \label{eq:ert}
\end{align}

From Equation \eqref{eq:ertp1} we have
\begin{equation}
  \E{R_t} = \{ \phi_{t - 1} + (1 - \gamma_t) \psi_{t - 1}\} \E{R_{t - 1}}, \nonumber
\end{equation}
and via substitution of $\E{R_{t - 1}}$ using \eqref{eq:ert}, we can
specify $\E{R_t}$ in terms of $\E{N_t}$:
\begin{equation}
  \E{R_t} = \frac{\{ \phi_{t - 1} + (1 - \gamma_t) \psi_{t - 1}\}
    \E{N_t}}{\phi_{t - 1} + \psi_{t - 1}}. \label{eq:ert2}
\end{equation}

By definition, each individual in the population in the first occasion
is a resident with probability $(1 - \gamma_1)$, so we also have
\begin{equation}
\E{R_1} = (1 - \gamma_1) \E{N_1}. \label{eq:er1}
\end{equation}
For our standard model, we specified the expected population size on
occasion $t + 1$ in terms of the expected population size on occasion
$t$ in Equation \eqref{eq:e1}. We can now achieve something similar
here. For the second occasion, we have
$\E{N_2} = (\phi_1 + \psi_1) \E{R_1}$ from \eqref{eq:entp1}, and
substituting in \eqref{eq:er1} gives
\begin{equation}
  \E{N_2}  = (\phi_1 + \psi_1) (1 - \gamma_1) \E{N_1}. \label{eq:EN2t}
\end{equation}

For subsequent occasions ($t = 2, \ldots$), we have $\E{N_{t + 1}} =
(\phi_t + \psi_t) \E{R_t}$ from \eqref{eq:entp1}, and substituting in
\eqref{eq:ert2} gives
\begin{equation}
  \E{N_{t + 1}} = \frac{(\phi_t + \psi_t) \{ \phi_{t - 1} + (1 - \gamma_t) \psi_{t - 1}\}
    \E{N_t}}{\phi_{t - 1} + \psi_{t - 1}}. \label{eq:entp1n1}
\end{equation}

Similarly to Equation \eqref{eq:e2} for standard models, we can also
express $E(N_{t + 1})$ in terms of the superpopulation size and
probabilities of entry, where
\begin{align}
  \E{N_{t + 1}} &= \phi_t \E{R_t} + p_{e, t + 1}M \\
                &= \frac{\phi_t \{ \phi_{t - 1} + (1 - \gamma_t) \psi_{t - 1}\}
                  \E{N_t}}{\phi_{t - 1} + \psi_{t - 1}} + p_{e, t + 1}M. \label{eq:entp1n2}
\end{align}

We can equate \eqref{eq:entp1n1} and \eqref{eq:entp1n2} to develop an
expression for the probabilities of entries as follows:
\begin{align}
  \frac{(\phi_t + \psi_t) \{ \phi_{t - 1} + (1 - \gamma_t) \psi_{t - 1}\}
    \E{N_t}}{\phi_{t - 1} + \psi_{t - 1}} &= \frac{\phi_t \{ \phi_{t - 1} + (1 - \gamma_t) \psi_{t - 1}\}
                                            \E{N_t}}{\phi_{t - 1} + \psi_{t - 1}} + p_{e, t + 1}M \\
  p_{e, t + 1} &= \frac{\psi_t \{ \phi_{t - 1} + (1 - \gamma_t)\psi_{t - 1} \} \E{N_t}}{M(\phi_{t - 1} + \psi_{t - 1})}. \label{eq:petp1t}
\end{align}
This expression applies to $p_{e, t + 1}$ for $t = 2, 3, \cdots$.  For
$t = 1$, corresponding to $p_{e, 2}$, we can equate \eqref{eq:EN2t}
with
$\E{N_2} = \phi_1\E{R_1} + p_{e, 2} M = \phi_1 (1 - \gamma_1) \E{N_1}
+ p_{e, 2}M$ to give
\begin{align}
  (\phi_1 + \psi_1) (1 - \gamma_1) \E{N_1} &= \phi_1 (1 - \gamma_1) \E{N_1}
                                             + p_{e, 2}M \\
  p_{e, 2} &= \frac{\psi_1 (1 - \gamma_1) \E{N_1}}{M}, \label{eq:pe2t}
\end{align}
As with the standard models, we can
\begin{enumerate}
\item initialise $p_{e, 1}$ at any
  value,
\item calculate $\E{N_1} = p_{e, 1} M$,
\item calculate $p_{e, 2}$
  from \eqref{eq:pe2t},
\item iteratively calculate the remaining
$\E{N_t}$ and $p_{e, t}$ terms using \eqref{eq:entp1n2} and
\eqref{eq:petp1t}, and
\item rescale the probabilities of entry
  calculated from the preceding steps so that they sum to one.
\end{enumerate}
We can then calculate the standard POPAN model likelihood using these
probabilities of entry.


\bibliographystyle{apalike} \bibliography{refs}

\end{document}
